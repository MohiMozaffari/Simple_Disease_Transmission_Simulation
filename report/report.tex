\documentclass[12pt,onecolumn,a4paper]{article}
\usepackage{hyperref}
\usepackage{hypcap}
\usepackage{graphicx}
\usepackage{listings}
\usepackage{xcolor}
\usepackage{csvsimple}
\usepackage{geometry}
\usepackage{float}

\geometry{
a4paper,
total={170mm,247mm},
left=20mm,
top=30mm,
}
\usepackage{tcolorbox}

% Define a custom style for the box
\tcbset{
    mybox/.style={
        colframe=magenta!90!black,
        colback=magenta!10,
        sharp corners,
        boxrule=1pt,
        fonttitle=\bfseries,
        title=خروجی کد,
        coltitle=black,
    }
}
\usepackage{paralist}
\usepackage{graphicx,subfigure,wrapfig}
\usepackage{mathalpha,fontspec,unicode-math}
\usepackage[fontsizescale = 1.1]{xepersian}

\setdigitfont{XB Yas}
\setmathfont[Scale= 1.2]{Latin Modern Math}
\settextfont[Scale= 1]{XB Yas}
\setlatintextfont[Scale = 1]{Latin Modern Roman}


\definecolor{codegreen}{rgb}{0,0.6,0}
\definecolor{codegray}{rgb}{0.5,0.5,0.5}
\definecolor{codepurple}{rgb}{0.58,0,0.82}
\definecolor{backcolour}{rgb}{0.95,0.95,0.92}

\lstdefinestyle{mystyle}{
    backgroundcolor=\color{backcolour},   
    commentstyle=\color{codegreen},
    keywordstyle=\color{magenta},
    numberstyle=\tiny\color{codegray},
    stringstyle=\color{codepurple},
    basicstyle=\ttfamily\footnotesize,
    breakatwhitespace=false,         
    breaklines=true,                 
    captionpos=b,                    
    keepspaces=true,                 
    numbers=left,                    
    numbersep=5pt,                  
    showspaces=false,                
    showstringspaces=false,
    showtabs=false,                  
    tabsize=2
}

\lstset{style=mystyle}


\setdigitfont{XB Yas}
\setmathfont[Scale= 1.2]{Latin Modern Math}
\settextfont[Scale= 1]{XB Yas}
\setlatintextfont[Scale = 1.2]{Latin Modern Roman}
\linespread{1.5}


\begin{document}
\begin{figure}[t]   
    \centerline{
        \includegraphics[height =5 cm]{Sbu-logo.svg.png}}
\end{figure}
\begin{center}
دانشگاه شهید بهشتی

دانشکدهٔ فیزیک

\vspace{30pt}

\vspace{12pt}
\Huge
\textbf{
    گزارش کار برنامه شبیه‌سازی انتقال بیماری
}

\LARGE

\vspace{100pt}

\textbf{
محدثه مظفری}




\vspace{30pt}
\large
آبان
۱۴۰۲

\end{center}


\pagebreak

\section{مقدمه}
این برنامه یک مدل ساده از انتقال بیماری در یک جمعیت با استفاده از یک حلقه ارائه می‌دهد. برنامه با استفاده از متغیرهای
\lr{ALPHA}
،
\lr{BETA}
،
\lr{$NUM\_ILLNESS$}
و
\lr{STEPS}
تنظیمات اجرای شبیه‌سازی را مشخص می‌کند.

با گذر زمان
\lr{(STEPS)}
، انتقال بیماری بر اساس 
\lr{ALPHA}
 و 
 \lr{BETA}
اعمال می‌شود.
اگر یک فرد بیمار باشد ، با احتمال 
\lr{ALPHA}
به افراد همسایه بیماری منتقل می‌شود.
با احتمال 
\lr{BETA}
فرد بیمار به حالت 2 (بهبود یافته) تغییر می‌کند.
حالت نهایی جمعیت پس از گذر از تمام مراحل نمایش داده می‌شود.

\section{نتایج شبیه‌سازی}

برای شرایط اولیه
\lr{N = 10}
،
\lr{ALPHA = 0.3}
،
\lr{BETA = 0.2}
،
\lr{$NUM\_ILLNESS$ = 2}
و
\lr{STEPS = 5000}
نتیجه شبیه‌سازی به صورت زیر می‌باشد:

\begin{tcolorbox}[mybox]
        \LTR
        \lr{0101000000}

        \lr{2222222222}
\end{tcolorbox}
\RTL

حالت اول شرایط اولیه را نمایش می‌دهد.
که اعداد ۱ نشانگر افراد بیمار و اعداد ۰ نشانگر افراد سالم است.

حالت دوم، حالت نهایی سیستم را نشان می‌دهد
که همهٔ افراد بهبود‌یافته اند.


برای شرایط اولیه
\lr{N = 10}
،
\lr{ALPHA = 0.2}
،
\lr{BETA = 0.2}
،
\lr{$NUM\_ILLNESS$ = 2}
و
\lr{STEPS = 5000}
نتیجه شبیه‌سازی به صورت زیر می‌باشد:

\begin{tcolorbox}[mybox]
        \LTR
        \lr{0000100010}

        \lr{2222200022}
\end{tcolorbox}
\RTL

همان‌طور که مشاهده می‌شود؛ با تغییر مقدار 
\lr{ALPHA}
حالت نهایی سیستم تغییر می‌کند.
زیرا ممکن است افراد بهبود‌یافته باشند و دیگر نتوانند همه سیستم را بیمار کنند.


\section{کد شبیه‌سازی}
\LTR
\lstinputlisting[language=C++]{C:/Users/Asus/Documents/Programming/Disorder/main/main.cpp}
\RTL



\end{document}